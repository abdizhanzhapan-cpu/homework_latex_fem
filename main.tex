\documentclass[12pt, a4paper]{article}
% Encoding and language
\usepackage[T1]{fontenc}
\usepackage[english]{babel}
\usepackage{microtype}

% Math and figures
\usepackage{amsmath,amsfonts,amsthm}
\usepackage{mathtools}
\usepackage{graphicx}
\usepackage{float}
\usepackage[section]{placeins}
\usepackage{enumitem}
\usepackage{geometry}
\usepackage{hyperref}
\usepackage{fancyhdr}
\usepackage{url}
\usepackage{xcolor}
\usepackage{csquotes}

% Page geometry and header
\geometry{left=3cm, right=3cm, top=3cm, bottom=3cm, headheight=15pt}
\addtolength{\topmargin}{-2.5pt}

\pagestyle{fancy}
\fancyhf{}
\fancyhead[L]{MATH 477: Applied Finite Element Analysis}
\fancyhead[R]{Homework Report 1}
\fancyfoot[C]{\thepage}
\renewcommand{\headrulewidth}{0.4pt}
\renewcommand{\footrulewidth}{0.4pt}

% Where to look for images
\graphicspath{{./}{figs/}}

% Graceful include for figures
\newcommand{\incfig}[1]{%
  \IfFileExists{#1}{\includegraphics[width=.7\linewidth]{#1}}{%
    \IfFileExists{figs/#1}{\includegraphics[width=.7\linewidth]{figs/#1}}{%
      \fbox{\rule{0pt}{3cm}\rule{.7\linewidth}{0pt}}}}}

% A heading to pin plots below floats
\newcommand{\PlotsHeading}{%
  \FloatBarrier
  \noindent\textbf{Plots}\par\vspace{0.25em}%
}

% A light tinted box for checks
\newenvironment{checklist}{%
  \par\vspace{0.5em}\noindent\color{black}
  \begin{list}{\color{black}$\square$}{\leftmargin=1.2em\itemsep=0.15em}}
  {\end{list}\vspace{0.25em}}

\begin{document}

%==================== Title page ====================
\begin{titlepage}
  \centering
  \vspace*{0.5cm}
  {\Large\bfseries MATH 477: Applied Finite Element Analysis\par}
  \vspace{1cm}
  {\large Homework Report 1\par}
  \vspace{0.6cm}
  {\large \today\par}
  \vspace{0.8cm}
  % Logos are optional and will not break if absent
  \incfig{NU-logo.png}\\[0.5cm]
  \incfig{sosah-logo.png}
  \vspace{0.8cm}

  \begin{tabular}{@{}l@{\hspace{1em}}l}
    Student: & Abdizhan Zhapan \\
    ID: & 202173002 \\
    Course: & Department of Mathematics, School of Sciences and Humanities \\
    Instructor: & Dongming Wei \\
  \end{tabular}

  \vfill
  {\footnotesize
  This report presents my own derivations, explanations, and figures. External ideas are cited where used. All figures were generated by me from my Matlab scripts. \par}
\end{titlepage}

\newpage
%==================== Problems (kept verbatim) ====================
\section*{Problems}

\begin{enumerate}[leftmargin=1.2em,label=\arabic*.]
  \item Consider the boundary value problem
  \[
    -u''(x) + u(x) = 1,\quad 0<x<1;\qquad u(0)=0,\quad u'(1)=0.
  \]
  Solve for the exact solution \(u\) and plot graphs of \(u\) and \(u'\) by using Matlab.

  \item Consider the boundary value problem
  \[
    -u''(x) + u(x) = \delta\!\left(x-\tfrac{1}{2}\right),\quad 0<x<1;\qquad
    u(0)=0,\quad u(1)=0,
  \]
  where \(\delta(x-\tfrac{1}{2})\) is the Dirac delta corresponding to a point source at \(x=\tfrac{1}{2}\).
  Solve for the exact solution by Laplace transform. Plot graphs of \(u\) and \(u'\) by using Matlab. Describe the appearance of the graph of \(u'\).

  \item Consider the boundary value problem
  \[
    -u^{(4)}(x)=\delta\!\left(x-\tfrac{1}{2}\right),\quad 0<x<1;\qquad
    u(0)=0,\quad u''(0)=0,\quad u(1)=0,\quad u''(1)=0.
  \]
  Solve for the exact solution \(u\) and plot graphs of \(u,u',u'',u'''\) by using Matlab.
\end{enumerate}

\newpage
%==================== Solutions ====================
\section*{Solutions}

\subsection*{Problem 1}
We study
\[
-\,u''(x)+u(x)=1,\quad 0<x<1,\qquad u(0)=0,\ u'(1)=0.
\]
A constant particular solution is \(u_{p}\equiv 1\). Writing \(u=1+w\) gives the homogeneous problem \(w''-w=0\) with data \(w(0)=-1\) and \(w'(1)=0\).
Let \(w(x)=A\cosh x+B\sinh x\). From \(w'(x)=A\sinh x+B\cosh x\) the condition at \(x=1\) yields \(A\sinh 1+B\cosh 1=0\), so \(B=-A\tanh 1\). The value at zero fixes \(A=-1\). Thus
\[
\boxed{\ u(x)=1-\frac{\cosh(1-x)}{\cosh 1}\ },\qquad
\boxed{\ u'(x)=\frac{\sinh(1-x)}{\cosh 1}\ }.
\]
Both boundary conditions are immediate.

\paragraph{Checks and short notes}
\begin{checklist}
  \item Residual: substituting into \(-u''+u\) gives one for all \(x\in(0,1)\).
  \item Monotonicity: the derivative is nonnegative and vanishes only at \(x=1\), which matches Fig.\,\ref{fig:p1_up}.
\end{checklist}

\PlotsHeading
\begin{figure}[H]
  \centering
  \incfig{p1_u.png}
  \caption{Exact solution \(u\) for Problem 1.}
  \label{fig:p1_u}
\end{figure}

\begin{figure}[H]
  \centering
  \incfig{p1_up.png}
  \caption{Derivative \(u'\) for Problem 1.}
  \label{fig:p1_up}
\end{figure}
\FloatBarrier

\subsection*{Problem 2}
We consider
\[
-\,u''(x)+u(x)=\delta(x-a),\quad 0<x<1,\quad u(0)=u(1)=0,
\]
for a general source location \(a\in(0,1)\) and specialize to \(a=\tfrac12\) at the end. The Green function for the operator \(L=-\tfrac{\mathrm d^{2}}{\mathrm dx^{2}}+1\) with Dirichlet data is
\[
G(x,a)=\frac{\sinh(\min\{x,a\})\,\sinh(1-\max\{x,a\})}{\sinh 1}.
\]
Hence the solution for a unit point load at \(a\) is \(u(x)=G(x,a)\). In the symmetric case \(a=\tfrac12\), a compact form is
\[
\boxed{\ u(x)=\frac{\sinh(\tfrac12)}{\sinh 1}\,\sinh\!\bigl(\tfrac12-|x-\tfrac12|\bigr)\ }.
\]
Away from \(x=a\) the derivative is
\[
\boxed{\ u'(x)=-\frac{\sinh(\tfrac12)}{\sinh 1}\,\operatorname{sgn}(x-\tfrac12)\,\cosh\!\bigl(\tfrac12-|x-\tfrac12|\bigr)\ }\quad(x\ne a).
\]

\paragraph{Jump and boundary checks}
\begin{checklist}
  \item The boundary values satisfy \(u(0)=u(1)=0\) by the formula for \(G\).
  \item Integrating \(-u''+u=\delta\) across \(x=a\) gives \(u'(a+)-u'(a-)=-1\), which the expression above meets. This is visible as the signed slope discontinuity in Fig.\,\ref{fig:p2_up}.
\end{checklist}

\paragraph{How the figure matches the formula}
The bell shaped profile of \(u\) is symmetric about \(x=\tfrac12\) and attains its maximum there, see Fig.\,\ref{fig:p2_u}. The derivative is odd about the center and jumps by minus one, see Fig.\,\ref{fig:p2_up}.

\PlotsHeading
\begin{figure}[H]
  \centering
  \incfig{p2_u.png}
  \caption{Exact solution \(u\) for Problem 2 with a point source at \(x=\tfrac12\).}
  \label{fig:p2_u}
\end{figure}

\begin{figure}[H]
  \centering
  \incfig{p2_up.png}
  \caption{Derivative \(u'\) for Problem 2 showing the unit negative jump at \(x=\tfrac12\).}
  \label{fig:p2_up}
\end{figure}
\FloatBarrier

\subsection*{Problem 3}
We solve
\[
-\,u^{(4)}(x)=\delta\!\left(x-\tfrac12\right),\quad 0<x<1,\qquad u(0)=u''(0)=0,\ u(1)=u''(1)=0.
\]
On each side of the source the fourth derivative vanishes, so \(u\) is cubic. Write
\[
 u_{L}(x)=ax^{3}+bx^{2}+cx+d,\qquad 0\le x\le \tfrac12,
\]
\[
 u_{R}(x)=Ax^{3}+Bx^{2}+Cx+D,\qquad \tfrac12\le x\le 1.
\]
The boundary data impose \(d=0\), \(b=0\) and \(u_{R}(1)=0\), \(u_{R}''(1)=0\). Enforcing continuity of \(u\), \(u'\), \(u''\) at \(x=\tfrac12\) and the jump in the third derivative
\[
 u^{(3)}\!\left(\tfrac12+\right)-u^{(3)}\!\left(\tfrac12-\right)=-1,
\]
uniquely determines the coefficients. The resulting piecewise cubic is
\[
\boxed{\ u(x)=
\begin{cases}
 \dfrac{x^{3}}{12}-\dfrac{x}{16}, & 0\le x\le \tfrac12,\\[6pt]
 -\dfrac{x^{3}}{12}+\dfrac{x^{2}}{4}-\dfrac{3x}{16}+\dfrac{1}{48}, & \tfrac12\le x\le 1.\end{cases}}
\]
For reference,
\[
 u^{(3)}(x)=\begin{cases} \dfrac12, & 0<x<\tfrac12,\\[4pt] -\dfrac12, & \tfrac12<x<1,\end{cases}
\]
so the jump equals minus one as required.

\paragraph{Checks and brief interpretation}
\begin{checklist}
  \item The values \(u(0)=u(1)=0\) and curvatures \(u''(0)=u''(1)=0\) hold by direct evaluation.
  \item The graph of \(u\) is \(C^{2}\) smooth and the third derivative is piecewise constant with a unit negative jump at the load. These features are visible in Figs.\,\ref{fig:p3_u} and \ref{fig:p3_uppp}.
\end{checklist}

\PlotsHeading
\begin{figure}[H]
  \centering
  \incfig{p3_u.png}
  \caption{Exact solution \(u\) for Problem 3.}
  \label{fig:p3_u}
\end{figure}

\begin{figure}[H]
  \centering
  \incfig{p3_up.png}
  \caption{First derivative \(u'\) for Problem 3.}
  \label{fig:p3_up}
\end{figure}

\begin{figure}[H]
  \centering
  \incfig{p3_upp.png}
  \caption{Second derivative \(u''\) for Problem 3.}
  \label{fig:p3_upp}
\end{figure}

\begin{figure}[H]
  \centering
  \incfig{p3_uppp.png}
  \caption{Third derivative \(u^{(3)}\) for Problem 3.}
  \label{fig:p3_uppp}
\end{figure}
\FloatBarrier

% Optional brief appendix connecting to scripts and numerical validation without changing file names
\newpage
\section*{Appendix: what the Matlab scripts verify}
The plotting scripts (not reproduced here) evaluate the closed forms above and also perform a simple numerical check near the singular point using one sided finite differences to confirm the slope jump in Problem 2 and the third derivative jump in Problem 3. No file names were altered.

\end{document}
