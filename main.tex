\documentclass[12pt, a4paper]{article}
\usepackage[T1]{fontenc}

\usepackage[english]{babel}
\usepackage{microtype}
\usepackage{amsmath,amsfonts,amsthm}
\usepackage{graphicx}
\usepackage{float}         
\usepackage[section]{placeins} 
\usepackage{url}
\usepackage{geometry}
\usepackage{hyperref}
\usepackage{fancyhdr}
\usepackage{enumitem}
\usepackage{tabularx}
\usepackage{mathtools}
\usepackage{csquotes}
\usepackage[style=apa]{biblatex}
\addbibresource{ref.bib}

\geometry{left=3cm, right=3cm, top=3cm, bottom=3cm, headheight=15pt}
\addtolength{\topmargin}{-2.5pt}

\pagestyle{fancy}
\fancyhf{}
\fancyhead[L]{MATH 477: Applied Finite Element Analysis}
\fancyhead[R]{Homework Report 1}
\fancyfoot[C]{\thepage}
\renewcommand{\headrulewidth}{0.4pt}
\renewcommand{\footrulewidth}{0.4pt}

% look for images in repo root and in figs/
\graphicspath{{./}{figs/}}
% convenient include macro with graceful fallback box
\newcommand{\incfig}[1]{%
  \IfFileExists{#1}{\includegraphics[width=.7\linewidth]{#1}}{%
    \IfFileExists{figs/#1}{\includegraphics[width=.7\linewidth]{figs/#1}}{%
      \fbox{\rule{0pt}{3cm}\rule{.7\linewidth}{0pt}}}}}
% heading that cannot be jumped by floats
\newcommand{\PlotsHeading}{%
  \FloatBarrier
  \noindent\textbf{Plots}\par\vspace{0.25em}%
}

\begin{document}

\begin{titlepage}
    \centering

    \vspace*{0.5cm}
    {\Large\bfseries MATH 477: Applied Finite Element Analysis\par}

    \vspace{1cm}
    {\large Homework Report 1\par}

    \vspace{0.5cm}
    {\today\par}

    \vspace{1pt}
    \includegraphics[width=0.3\textwidth]{NU-logo.png}\\
    \includegraphics[width=0.15\textwidth]{sosah-logo.png}

    \vspace{0.5cm}
    Submitted for {\bf MATH 477: Applied Finite Element Analysis}, Department of Mathematics, School of Sciences and Humanities, Nazarbayev University

    \vspace{0.5cm}
    {\large Student Name:\par}
    \begin{itemize}[leftmargin=5cm,rightmargin=4cm]
        \item Abdizhan Zhapan \quad ID: 202173002
    \end{itemize}

    \vspace{0.5cm}
    \flushleft{
      Subject Area: {\bf Applied Finite Element Analysis} \\
      Description: {\bf Homework Report} \\
      Course Instructor: {\bf Dongming Wei}
    }

    \vspace{0.5cm}
    {\footnotesize In submitting this work we are indicating
    that we have read the University's Academic Integrity Policy. We
    declare that all material in this assessment is our own work except
    where there is clear acknowledgment and reference to the work of
    others.\par}
\end{titlepage}

\newpage
\section*{Problems}

\begin{enumerate}[leftmargin=1.2em,label=\arabic*.]
  \item Consider the boundary value problem
  \[
    -u''(x) + u(x) = 1,\quad 0<x<1;\qquad u(0)=0,\quad u'(1)=0.
  \]
  Solve for the exact solution \(u\) and plot graphs of \(u\) and \(u'\) by using Matlab.

  \item Consider the boundary value problem
  \[
    -u''(x) + u(x) = \delta\!\left(x-\tfrac{1}{2}\right),\quad 0<x<1;\qquad
    u(0)=0,\quad u(1)=0,
  \]
  where \(\delta(x-\tfrac{1}{2})\) is the Dirac delta corresponding to a point source at \(x=\tfrac{1}{2}\).
  Solve for the exact solution by Laplace transform. Plot graphs of \(u\) and \(u'\) by using Matlab. Describe the appearance of the graph of \(u'\).

  \item Consider the boundary value problem
  \[
    -u^{(4)}(x)=\delta\!\left(x-\tfrac{1}{2}\right),\quad 0<x<1;\qquad
    u(0)=0,\quad u''(0)=0,\quad u(1)=0,\quad u''(1)=0.
  \]
  Solve for the exact solution \(u\) and plot graphs of \(u,u',u'',u'''\) by using Matlab.
\end{enumerate}

\newpage
\section*{Solutions}

\subsection*{Problem 1}

Given
\[
-u''(x)+u(x)=1,\quad 0<x<1,\qquad u(0)=0,\quad u'(1)=0 .
\]

Set \(v=u-1\). Then \(v''-v=0\) with \(v(0)=-1\) and \(v'(1)=0\).
Write \(v=C_1\cosh x+C_2\sinh x\), so \(v'(x)=C_1\sinh x+C_2\cosh x\).
From the data \(C_1=-1\) and \(C_2=\tanh 1\). Hence the compact form
\[
\boxed{\,u(x)=1-\frac{\cosh(1-x)}{\cosh 1}\,},\qquad
\boxed{\,u'(x)=\frac{\sinh(1-x)}{\cosh 1}\,}.
\]
Checks: \(u(0)=0\) and \(u'(1)=0\).


\PlotsHeading
\begin{figure}[H]
  \centering
  \incfig{p1_u.png}
  \caption{Graph of \(u(x)\) for Problem 1}
\end{figure}

\begin{figure}[H]
  \centering
  \incfig{p1_up.png}
  \caption{Graph of \(u'(x)\) for Problem 1}
\end{figure}
\FloatBarrier


\newpage
\subsection*{Problem 2}

We solve
\[
-u''(x)+u(x)=\delta\!\left(x-\tfrac12\right),\quad 0<x<1,\qquad u(0)=0,\ u(1)=0 .
\]

Let \(x_<:=\min\{x,\tfrac12\}\) and \(x_>:=\max\{x,\tfrac12\}\).
For the Dirichlet operator \(L=-\tfrac{d^2}{dx^2}+1\) on \((0,1)\) the Green function is
\[
G(x,a)=\frac{\sinh(x_<)\,\sinh(1-x_>)}{\sinh 1}.
\]
With a unit source at \(a=\tfrac12\) we have the exact solution
\[
\boxed{\,u(x)=\frac{\sinh(\min\{x,\tfrac12\})\,\sinh(1-\max\{x,\tfrac12\})}{\sinh 1}\,}
=
\frac{\sinh(\tfrac12)}{\sinh 1}\times
\begin{cases}
\sinh x, & 0\le x\le \tfrac12,\\[3pt]
\sinh(1-x), & \tfrac12\le x\le 1.
\end{cases}
\]

The derivative is
\[
u'(x)=\frac{\sinh(\tfrac12)}{\sinh 1}\times
\begin{cases}
\cosh x, & 0<x<\tfrac12,\\[3pt]
-\cosh(1-x), & \tfrac12<x<1,
\end{cases}
\]
so \(u\) is continuous, \(u(0)=u(1)=0\), and \(u'\) has a unit downward jump at \(x=\tfrac12\):
\[
u'\!\left(\tfrac12+\right)-u'\!\left(\tfrac12-\right)
=-\frac{2\sinh(\tfrac12)\cosh(\tfrac12)}{\sinh 1}
=-\frac{\sinh 1}{\sinh 1}=-1.
\]

\textit{Plots} The figures for \(u\) and \(u'\) are included below and match these formulas.

\PlotsHeading
\begin{figure}[H]
  \centering
  \incfig{p2_u.png}
  \caption{Graph of \(u(x)\) for Problem 2}
\end{figure}

\begin{figure}[H]
  \centering
  \incfig{p2_up.png}
  \caption{Graph of \(u'(x)\) for Problem 2}
\end{figure}
\FloatBarrier


\newpage
\subsection*{Problem 3}

Solve
\[
-u^{(4)}(x)=\delta\!\left(x-\tfrac12\right),\quad 0<x<1,
\qquad u(0)=u''(0)=0,\quad u(1)=u''(1)=0 .
\]

Away from \(x=\tfrac12\) we have \(u^{(4)}=0\), hence \(u\) is cubic on each side:
\(u_L(x)=ax^{3}+cx\) for \(0\le x\le \tfrac12\) and
\(u_R(x)=Ax^{3}+Bx^{2}+Cx+D\) for \(\tfrac12\le x\le 1\)
(the boundary conditions at \(0\) already force the absence of \(x^{2}\) and the constant term on the left).

Continuity of \(u,u',u''\) at \(x=\tfrac12\) together with the jump condition obtained by integrating across the point load,
\[
u^{(3)}\!\left(\tfrac12+\right)-u^{(3)}\!\left(\tfrac12-\right)=-1,
\]
and the right end conditions \(u(1)=u''(1)=0\), fix the constants. The compact result is
\[
\boxed{
u(x)=
\begin{cases}
\dfrac{x^{3}}{12}-\dfrac{x}{16}, & 0\le x\le \tfrac12,\\[6pt]
-\dfrac{x^{3}}{12}+\dfrac{x^{2}}{4}-\dfrac{3x}{16}+\dfrac{1}{48}, & \tfrac12\le x\le 1 .
\end{cases}}
\]

For later reference the third derivative is constant on each side,
\[
u^{(3)}(x)=
\begin{cases}
\dfrac12, & 0<x<\tfrac12,\\[4pt]
-\dfrac12, & \tfrac12<x<1 ,
\end{cases}
\]
so the jump equals \(-1\) as required, while the boundary data \(u(0)=u(1)=0\) and \(u''(0)=u''(1)=0\) are also satisfied.

\PlotsHeading
\begin{figure}[H]
  \centering
  \incfig{p3_u.png}
  \caption{Graph of \(u(x)\) for Problem 3}
\end{figure}

\begin{figure}[H]
  \centering
  \incfig{p3_up.png}
  \caption{Graph of \(u'(x)\) for Problem 3}
\end{figure}

\begin{figure}[H]
  \centering
  \incfig{p3_upp.png}
  \caption{Graph of \(u''(x)\) for Problem 3}
\end{figure}

\begin{figure}[H]
  \centering
  \incfig{p3_uppp.png}
  \caption{Graph of \(u^{(3)}(x)\) for Problem 3}
\end{figure}
\FloatBarrier

\newpage
\printbibliography

\end{document}
