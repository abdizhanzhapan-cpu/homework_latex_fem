% ========= in Problems =========
\item A uniform Euler Bernoulli beam of length $L$ and mass $m$ is clamped at both ends.
A vertical point load $P$ is applied at $x=L/3$.
Let $\phi(x)$ denote the small transverse deflection.
Derive from first principles the boundary value model for $\phi$ that includes the self weight of the beam.

% ========= in Solutions =========
\subsection*{Problem 1a  derivation of the model}

\textbf{Kinematics and material law.}
For small deflections the curvature of the neutral axis is
\[
\kappa(x)\approx \dv[2]{\phi}{x}.
\]
For a linear elastic prismatic beam with Young modulus $E$ and second moment of area $I$ the bending moment is related to curvature by
\[
M(x)=-EI\,\kappa(x)=-EI\,\phi''(x).
\]

\textbf{Static balance.}
Let $Q(x)$ be the shear force and $q(x)$ the distributed transverse load per unit length taken positive downward.
Classical relations give
\[
Q=\dv{M}{x},\qquad q=\dv{Q}{x}=\dv[2]{M}{x}.
\]
Substituting the constitutive relation yields the field equation
\[
EI\,\phi^{(4)}(x)=q(x).
\]

\textbf{Loads in the present setting.}
There are two sources
a concentrated force of magnitude $P$ at $x=L/3$ and the uniform self weight.
With total mass $m$ and gravity $g$ the constant load per unit length is
\[
\Omega=\frac{mg}{L}.
\]
Hence the total load density is
\[
q(x)=P\,\delta\!\left(x-\frac{L}{3}\right)+\Omega
= P\,\delta\!\left(x-\frac{L}{3}\right)+\frac{mg}{L}.
\]
Therefore
\[
EI\,\phi^{(4)}(x)=P\,\delta\!\left(x-\frac{L}{3}\right)+\frac{mg}{L}.
\]

\textbf{Boundary conditions.}
Both ends are clamped so deflection and rotation vanish
\[
\phi(0)=0,\qquad \phi'(0)=0,\qquad \phi(L)=0,\qquad \phi'(L)=0.
\]

\textbf{Jump implied by the point load.}
Integrating the field equation over a small interval centered at $x=L/3$ gives the shear jump
\[
\phi^{(3)}\!\left(\Bigl(\tfrac{L}{3}\Bigr)^{+}\right)
-
\phi^{(3)}\!\left(\Bigl(\tfrac{L}{3}\Bigr)^{-}\right)
=\frac{P}{EI}.
\]

\textbf{Final model.}
With the notation above the boundary value problem is
\[
\boxed{\;
EI\,\phi^{(4)}(x)=P\,\delta\!\left(x-\frac{L}{3}\right)+\frac{mg}{L},
\qquad
\phi(0)=\phi'(0)=0,\quad \phi(L)=\phi'(L)=0
\;}
\]
which matches the statement of the assignment after the change of variable $\phi\leftrightarrow v$ and $\Omega=mg/L$.

