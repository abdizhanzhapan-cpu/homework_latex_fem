\documentclass[12pt, a4paper]{article}
\usepackage[T1]{fontenc}

\usepackage[english]{babel}
\usepackage{microtype}
\usepackage{amsmath,amsfonts,amsthm}
\usepackage{graphicx}
\usepackage{float}         
\usepackage[section]{placeins} 
\usepackage{url}
\usepackage{geometry}
\usepackage{hyperref}
\usepackage{fancyhdr}
\usepackage{enumitem}
\usepackage{tabularx}
\usepackage{mathtools}
\usepackage{csquotes}
\usepackage[style=apa]{biblatex}
\addbibresource{ref.bib}

\geometry{left=3cm, right=3cm, top=3cm, bottom=3cm, headheight=15pt}
\addtolength{\topmargin}{-2.5pt}

\pagestyle{fancy}
\fancyhf{}
\fancyhead[L]{MATH 477: Applied Finite Element Analysis}
\fancyhead[R]{Homework Report 1}
\fancyfoot[C]{\thepage}
\renewcommand{\headrulewidth}{0.4pt}
\renewcommand{\footrulewidth}{0.4pt}

% look for images in repo root and in figs/
\graphicspath{{./}{figs/}}
% convenient include macro with graceful fallback box
\newcommand{\incfig}[1]{%
  \IfFileExists{#1}{\includegraphics[width=.7\linewidth]{#1}}{%
    \IfFileExists{figs/#1}{\includegraphics[width=.7\linewidth]{figs/#1}}{%
      \fbox{\rule{0pt}{3cm}\rule{.7\linewidth}{0pt}}}}}
% heading that cannot be jumped by floats
\newcommand{\PlotsHeading}{%
  \FloatBarrier
  \noindent\textbf{Plots}\par\vspace{0.25em}%
}

\begin{document}

\begin{titlepage}
    \centering

    \vspace*{0.5cm}
    {\Large\bfseries MATH 477: Applied Finite Element Analysis\par}

    \vspace{1cm}
    {\large Homework Report 1\par}

    \vspace{0.5cm}
    {\today\par}

    \vspace{1pt}
    \includegraphics[width=0.3\textwidth]{NU-logo.png}\\
    \includegraphics[width=0.15\textwidth]{sosah-logo.png}

    \vspace{0.5cm}
    Submitted for {\bf MATH 477: Applied Finite Element Analysis}, Department of Mathematics, School of Sciences and Humanities, Nazarbayev University

    \vspace{0.5cm}
    {\large Student Name:\par}
    \begin{itemize}[leftmargin=5cm,rightmargin=4cm]
        \item Abdizhan Zhapan \quad ID: 202173002
    \end{itemize}

    \vspace{0.5cm}
    \flushleft{
      Subject Area: {\bf Applied Finite Element Analysis} \\
      Description: {\bf Homework Report} \\
      Course Instructor: {\bf Dongming Wei}
    }

    \vspace{0.5cm}
    {\footnotesize In submitting this work we are indicating
    that we have read the University's Academic Integrity Policy. We
    declare that all material in this assessment is our own work except
    where there is clear acknowledgment and reference to the work of
    others.\par}
\end{titlepage}

\newpage
\section*{Problems}

\begin{enumerate}[leftmargin=1.2em,label=\arabic*.]
  \item Consider the boundary value problem
  \[
    -u''(x) + u(x) = 1,\quad 0<x<1;\qquad u(0)=0,\quad u'(1)=0.
  \]
  Solve for the exact solution \(u\) and plot graphs of \(u\) and \(u'\) by using Matlab.

  \item Consider the boundary value problem
  \[
    -u''(x) + u(x) = \delta\!\left(x-\tfrac{1}{2}\right),\quad 0<x<1;\qquad
    u(0)=0,\quad u(1)=0,
  \]
  where \(\delta(x-\tfrac{1}{2})\) is the Dirac delta corresponding to a point source at \(x=\tfrac{1}{2}\).
  Solve for the exact solution by Laplace transform. Plot graphs of \(u\) and \(u'\) by using Matlab. Describe the appearance of the graph of \(u'\).

  \item Consider the boundary value problem
  \[
    -u^{(4)}(x)=\delta\!\left(x-\tfrac{1}{2}\right),\quad 0<x<1;\qquad
    u(0)=0,\quad u''(0)=0,\quad u(1)=0,\quad u''(1)=0.
  \]
  Solve for the exact solution \(u\) and plot graphs of \(u,u',u'',u'''\) by using Matlab.
\end{enumerate}

\newpage
\section*{Solutions}

\subsection*{Problem 1}

Here I split the response into a constant part and a homogeneous remainder. Let the spatial label be \(\varrho\) and define \(\mathfrak{U}(\varrho)=u(\varrho)\). Peel off the constant particular piece by setting \(\Phi(\varrho)=\mathfrak{U}(\varrho)-1\). Then \(\Phi''-\Phi=0\) on \((0,1)\) with traces \(\Phi(0)=-1\) and \(\Phi'(1)=0\).

Write \(\Phi(\varrho)=\alpha\,\cosh\varrho+\beta\,\sinh\varrho\). The Neumann trace at \(1\) gives \(\alpha\sinh 1+\beta\cosh 1=0\), hence \(\beta=-\alpha\tanh 1\). The Dirichlet trace at \(0\) forces \(\alpha=-1\). Therefore the closed forms are
\[
\boxed{\,u(\varrho)=1-\frac{\cosh(1-\varrho)}{\cosh 1}\,},\qquad
\boxed{\,u'(\varrho)=\frac{\sinh(1-\varrho)}{\cosh 1}\,}.
\]
Quick checks: substituting back yields \(-u''+u\equiv 1\). The boundary data \(u(0)=0\) and \(u'(1)=0\) are immediate from the formulas.

\PlotsHeading
\begin{figure}[H]
  \centering
  \incfig{p1_u.png}
  \caption{Exact profile \(u\) for Problem 1}
\end{figure}

\begin{figure}[H]
  \centering
  \incfig{p1_up.png}
  \caption{Slope \(u'\) for Problem 1}
\end{figure}
\FloatBarrier


\newpage
\subsection*{Problem 2}

I solve once in general and then specialize to the mid point. Consider the operator \(L=-\frac{d^{2}}{dx^{2}}+1\) on \((0,1)\) with \(u(0)=u(1)=0\). Its Green kernel \(G(x,a)\) satisfies \(LG(\cdot,a)=\delta(\cdot-a)\) and vanishes at both ends. A compact expression that captures both sides at once is
\[
G(x,a)=\frac{\sinh(\min\{x,a\})\,\sinh(1-\max\{x,a\})}{\sinh 1}.
\]
Hence for a unit point source at \(a\) the solution is \(u(x)=G(x,a)\). At the symmetric location \(a=\tfrac12\) it is convenient to write everything through the distance \(d(x)=\bigl|x-\tfrac12\bigr|\):
\[
\boxed{\,u(x)=\frac{\sinh(\tfrac12)}{\sinh 1}\,\sinh\!\bigl(\tfrac12-d(x)\bigr)\,}.
\]
Away from the source the derivative can be written using the sign map,
\[
\boxed{\,u'(x)=-\frac{\sinh(\tfrac12)}{\sinh 1}\,\operatorname{sgn}\!\bigl(x-\tfrac12\bigr)\,
\cosh\!\bigl(\tfrac12-d(x)\bigr)\,}\quad(x\neq\tfrac12).
\]
Two standard verifications: the endpoints give \(u(0)=u(1)=0\); integrating \(Lu=\delta\) across \(x=\tfrac12\) yields the jump rule \(u'(\tfrac12+)-u'(\tfrac12-)=-1\), which the explicit \(u'\) satisfies.

\PlotsHeading
\begin{figure}[H]
  \centering
  \incfig{p2_u.png}
  \caption{Response \(u\) for a centered unit source}
\end{figure}

\begin{figure}[H]
  \centering
  \incfig{p2_up.png}
  \caption{Derivative \(u'\) showing the unit negative jump at the source}
\end{figure}
\FloatBarrier


\newpage
\subsection*{Problem 3}

Here the fourth derivative vanishes away from the load, so the field is cubic on each side of the split point. I parametrize the two pieces as
\[
u_{L}(x)=a\,x^{3}+b\,x^{2}+c\,x+d,\qquad 0\le x\le\tfrac12,
\]
\[
u_{R}(x)=A\,x^{3}+B\,x^{2}+C\,x+D,\qquad \tfrac12\le x\le 1.
\]
The boundary requirements \(u(0)=u''(0)=0\) give \(d=0\) and \(b=0\). The right end conditions \(u(1)=u''(1)=0\) constrain the right hand coefficients. At the interface \(x=\tfrac12\) I glue the pieces with continuity of \(u\), \(u'\) and \(u''\). The distributional equation enforces one final rule on the third derivative
\[
u^{(3)}\!\left(\tfrac12+\right)-u^{(3)}\!\left(\tfrac12-\right)=-1.
\]
Solving this linear system fixes all constants and yields the explicit cubic pair
\[
\boxed{\,u(x)=
\begin{cases}
\dfrac{x^{3}}{12}-\dfrac{x}{16}, & 0\le x\le \tfrac12,\\[6pt]
-\dfrac{x^{3}}{12}+\dfrac{x^{2}}{4}-\dfrac{3x}{16}+\dfrac{1}{48}, & \tfrac12\le x\le 1.
\end{cases}}
\]
For later reference the third derivative is piecewise constant,
\[
u^{(3)}(x)=
\begin{cases}
\dfrac12, & 0<x<\tfrac12,\\[4pt]
-\dfrac12, & \tfrac12<x<1,
\end{cases}
\]
so the jump has the correct magnitude and sign. One can check directly that \(u(0)=u(1)=0\) and \(u''(0)=u''(1)=0\).

\PlotsHeading
\begin{figure}[H]
  \centering
  \incfig{p3_u.png}
  \caption{Cubic piecewise profile \(u\)}
\end{figure}

\begin{figure}[H]
  \centering
  \incfig{p3_up.png}
  \caption{First derivative \(u'\)}
\end{figure}

\begin{figure}[H]
  \centering
  \incfig{p3_upp.png}
  \caption{Second derivative \(u''\)}
\end{figure}

\begin{figure}[H]
  \centering
  \incfig{p3_uppp.png}
  \caption{Third derivative \(u^{(3)}\)}
\end{figure}
\FloatBarrier

\newpage
\printbibliography

\end{document}
