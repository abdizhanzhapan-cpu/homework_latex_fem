\documentclass[12pt, a4paper]{article}
\usepackage[T1]{fontenc}

\usepackage[english]{babel}
\usepackage{microtype}
\usepackage{amsmath,amsfonts,amsthm}
\usepackage{graphicx}
\usepackage{url}
\usepackage{geometry}
\usepackage{hyperref}
\usepackage{fancyhdr}
\usepackage{enumitem}
\usepackage{tabularx}
\usepackage{mathtools}
\usepackage{csquotes}
\usepackage[style=apa]{biblatex}
\addbibresource{ref.bib}

\geometry{left=3cm, right=3cm, top=3cm, bottom=3cm, headheight=15pt}
\addtolength{\topmargin}{-2.5pt}

\pagestyle{fancy}
\fancyhf{}
\fancyhead[L]{MATH 477: Applied Finite Element Analysis}
\fancyhead[R]{Homework Report 1}
\fancyfoot[C]{\thepage}
\renewcommand{\headrulewidth}{0.4pt}
\renewcommand{\footrulewidth}{0.4pt}

% look for images in repo root and in figs/
\graphicspath{{./}{figs/}}
% convenient include macro with graceful fallback box
\newcommand{\incfig}[1]{%
  \IfFileExists{#1}{\includegraphics[width=.7\linewidth]{#1}}{%
    \IfFileExists{figs/#1}{\includegraphics[width=.7\linewidth]{figs/#1}}{%
      \fbox{\rule{0pt}{3cm}\rule{.7\linewidth}{0pt}}}}}

\begin{document}

\begin{titlepage}
    \centering

    \vspace*{0.5cm}
    {\Large\bfseries MATH 477: Applied Finite Element Analysis\par}

    \vspace{1cm}
    {\large Homework Report 1\par}

    \vspace{0.5cm}
    {\today\par}

    \vspace{1pt}
    \includegraphics[width=0.3\textwidth]{NU-logo.png}\\
    \includegraphics[width=0.15\textwidth]{sosah-logo.png}

    \vspace{0.5cm}
    Submitted for {\bf MATH 477: Applied Finite Element Analysis}, Department of Mathematics, School of Sciences and Humanities, Nazarbayev University

    \vspace{0.5cm}
    {\large Student Name:\par}
    \begin{itemize}[leftmargin=5cm,rightmargin=4cm]
        \item Abdizhan Zhapan \quad ID: 202173002
    \end{itemize}

    \vspace{0.5cm}
    \flushleft{
      Subject Area: {\bf Applied Finite Element Analysis} \\
      Description: {\bf Homework Report} \\
      Course Instructor: {\bf Dongming Wei}
    }

    \vspace{0.5cm}
    {\footnotesize In submitting this work we are indicating
    that we have read the University's Academic Integrity Policy. We
    declare that all material in this assessment is our own work except
    where there is clear acknowledgment and reference to the work of
    others.\par}
\end{titlepage}

\newpage
\section*{Problems}

\begin{enumerate}[leftmargin=1.2em,label=\arabic*.]
  \item Consider the boundary value problem
  \[
    -u''(x) + u(x) = 1,\quad 0<x<1;\qquad u(0)=0,\quad u'(1)=0.
  \]
  Solve for the exact solution \(u\) and plot graphs of \(u\) and \(u'\) by using Matlab.

  \item Consider the boundary value problem
  \[
    -u''(x) + u(x) = \delta\!\left(x-\tfrac{1}{2}\right),\quad 0<x<1;\qquad
    u(0)=0,\quad u(1)=0,
  \]
  where \(\delta(x-\tfrac{1}{2})\) is the Dirac delta corresponding to a point source at \(x=\tfrac{1}{2}\).
  Solve for the exact solution by Laplace transform. Plot graphs of \(u\) and \(u'\) by using Matlab. Describe the appearance of the graph of \(u'\).

  \item Consider the boundary value problem
  \[
    -u^{(4)}(x)=\delta\!\left(x-\tfrac{1}{2}\right),\quad 0<x<1;\qquad
    u(0)=0,\quad u''(0)=0,\quad u(1)=0,\quad u''(1)=0.
  \]
  Solve for the exact solution \(u\) and plot graphs of \(u,u',u'',u'''\) by using Matlab.
\end{enumerate}

\newpage
\section*{Solutions}

\subsection*{Problem 1}

We solve
\[
-u''(x)+u(x)=1,\quad 0<x<1,\qquad u(0)=0,\quad u'(1)=0.
\]
Equivalently,
\[
u''(x)-u(x)=-1.
\]
The homogeneous equation \(u''-u=0\) has general solution \(u_h(x)=A\cosh x+B\sinh x\).
A constant particular solution is \(u_p(x)=1\).
Hence
\[
u(x)=A\cosh x+B\sinh x+1.
\]
From \(u(0)=0\) we obtain \(A=-1\).
Since \(u'(x)=A\sinh x+B\cosh x\), the condition \(u'(1)=0\) gives
\[
-A\sinh 1 + B\cosh 1 = 0 \quad\Rightarrow\quad B=\tanh(1).
\]
Therefore the exact solution and its derivative are
\[
\boxed{\,u(x)=1-\cosh x+\tanh(1)\,\sinh x\,},\qquad
\boxed{\,u'(x)=-\sinh x+\tanh(1)\,\cosh x\,}.
\]

\paragraph{Plots}
\begin{figure}[h]
  \centering
  \incfig{p1_u.png}
  \caption{Graph of \(u(x)\) for Problem 1}
\end{figure}

\begin{figure}[h]
  \centering
  \incfig{p1_up.png}
  \caption{Graph of \(u'(x)\) for Problem 1}
\end{figure}

\newpage
\subsection*{Problem 2}

We solve
\[
-u''(x)+u(x)=\delta\!\left(x-\tfrac{1}{2}\right), \quad 0<x<1,
\qquad u(0)=0,\quad u(1)=0 .
\]

For the operator \(L u=-u''+u\) with homogeneous Dirichlet boundary conditions, the Green function is
\[
G(x,a)=
\begin{cases}
\dfrac{\sinh x\,\sinh(1-a)}{\sinh 1}, & 0\le x\le a,\\[6pt]
\dfrac{\sinh a\,\sinh(1-x)}{\sinh 1}, & a\le x\le 1,
\end{cases}
\]
where \(a\in(0,1)\). With the unit point load at \(a=\tfrac12\) the solution is \(u(x)=G(x,\tfrac12)\), i.e.
\[
\boxed{
u(x)=\frac{\sinh(\tfrac12)}{\sinh 1}
\begin{cases}
\sinh x, & 0\le x\le \tfrac12,\\[4pt]
\sinh(1-x), & \tfrac12\le x\le 1 .
\end{cases}}
\]
This \(u\) is continuous on \([0,1]\) and satisfies \(u(0)=u(1)=0\).

The derivative is piecewise
\[
u'(x)=\frac{\sinh(\tfrac12)}{\sinh 1}
\begin{cases}
\cosh x, & 0<x<\tfrac12,\\[4pt]
-\cosh(1-x), & \tfrac12<x<1 ,
\end{cases}
\]
and it has a unit downward jump at \(x=\tfrac12\):
\[
u'\!\left(\tfrac12+\right)-u'\!\left(\tfrac12-\right)
=-\,\frac{\sinh(\tfrac12)\cosh(\tfrac12)+\sinh(\tfrac12)\cosh(\tfrac12)}{\sinh 1}
=-1,
\]
which matches the distributional identity obtained by integrating
\(-u''+u=\delta\) across \(x=\tfrac12\).

\paragraph{Plots}
\begin{figure}[h]
  \centering
  \incfig{p2_u.png}
  \caption{Graph of \(u(x)\) for Problem 2}
\end{figure}

\begin{figure}[h]
  \centering
  \incfig{p2_up.png}
  \caption{Graph of \(u'(x)\) for Problem 2}
\end{figure}

\newpage
\subsection*{Problem 3}

We solve
\[
-u^{(4)}(x)=\delta\!\left(x-\tfrac{1}{2}\right),\quad 0<x<1,
\qquad u(0)=0,\quad u''(0)=0,\quad u(1)=0,\quad u''(1)=0 .
\]
On each side of \(a=\tfrac12\) the right hand side vanishes, so \(u\) is a cubic polynomial there. Enforcing
continuity of \(u,u',u''\) at \(x=\tfrac12\) and the jump condition
\[
u^{(3)}\!\left(\tfrac12+\right)-u^{(3)}\!\left(\tfrac12-\right)=-1,
\]
together with the four boundary conditions, yields the exact solution
\[
u(x)=
\begin{cases}
\dfrac{x^{3}}{12}-\dfrac{x}{16}, & 0\le x\le \tfrac12,\\[6pt]
-\dfrac{x^{3}}{12}+\dfrac{x^{2}}{4}-\dfrac{3x}{16}+\dfrac{1}{48}, & \tfrac12\le x\le 1,
\end{cases}
\]
with derivatives
\[
u'(x)=
\begin{cases}
\dfrac{x^{2}}{4}-\dfrac{1}{16}, & 0< x< \tfrac12,\\[6pt]
-\dfrac{x^{2}}{4}+\dfrac{x}{2}-\dfrac{3}{16}, & \tfrac12< x< 1,
\end{cases}
\qquad
u''(x)=
\begin{cases}
\dfrac{x}{2}, & 0\le x\le \tfrac12,\\[6pt]
-\dfrac{x}{2}+\dfrac{1}{2}, & \tfrac12\le x\le 1,
\end{cases}
\]
\[
u^{(3)}(x)=
\begin{cases}
\dfrac{1}{2}, & 0< x< \tfrac12,\\[6pt]
-\dfrac{1}{2}, & \tfrac12< x< 1,
\end{cases}
\]
so \(u,u',u''\) are continuous at \(x=\tfrac12\) while \(u^{(3)}\) has a unit downward jump \(-1\), which matches the equation.

\paragraph{Plots}
\begin{figure}[h]
  \centering
  \incfig{p3_u.png}
  \caption{Graph of \(u(x)\) for Problem 3}
\end{figure}

\begin{figure}[h]
  \centering
  \incfig{p3_up.png}
  \caption{Graph of \(u'(x)\) for Problem 3}
\end{figure}

\begin{figure}[h]
  \centering
  \incfig{p3_upp.png}
  \caption{Graph of \(u''(x)\) for Problem 3}
\end{figure}

\begin{figure}[h]
  \centering
  \incfig{p3_uppp.png}
  \caption{Graph of \(u^{(3)}(x)\) for Problem 3}
\end{figure}

\newpage
\printbibliography

\end{document}
